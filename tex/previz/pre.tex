\section*{上篇回顾}
%\section{\colorbox{head1}{\makebox[\hfill][l]{上篇回顾}}}

首先,我们简单的回顾一下上篇的内容,感兴趣的读者也可以参阅原链接。
Previz,中译可视化预览, 是指通过CG或者简单拍摄的方法粗略地展示影片内容的方法。根据不同的目的, Previz又可以分为
\begin{itemize}
	\item Pitch Previz,为了向投资人展示,以获得认可
	\item Tech Previz,按照故事板(story board)用CG或简单拍摄来确定布局、光照、摄像机位置焦距等信息
	\item Onset Previz,在实际拍摄过程中,实时的预览最终的合成效果,以及时判断是否需要补拍
	\item Post Previz,在后期制作之前,针对关键的或者不确定的一些特效镜头先做的一些大体效果,以最终确定是否需要
\end{itemize}

在这几种Previz中,Onset Previz最复杂,技术难度最高,而这也逐渐演变为另外一个技术, Virtual Production (虚拟拍摄)。VP专门是指同步虚拟摄像机和真实摄像机, 运动捕捉数据,实时的把虚拟场景和真实拍摄画面渲染出来的方法,

由于其他几种Previz都比较简单,接下来我们讨论的Previz也就是特指Onset Previz/Virtual Production。
