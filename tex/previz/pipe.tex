\section*{Previz拍摄流程是怎样的?}
首先便是虚拟场景的制作,包括环境,角色建模及骨架,纹理, 光照等,由于是Previz,所以模型和纹理等都不需要太高精度。

其次便是场景布局,由于摄像机运动信息的方法不同,场景布局有很大的区别。简单的,只要把绿屏和摄像机导轨搭建起来就好,复杂的需要在拍摄现场贴各种靶标,用于定位摄像机。

在实际的拍摄中,Previz系统会把几个通道采集的数据合成,并最终呈现在显示设备上,所有的参与制作的人员都可以看到,导演可以根据效果做出实时的判断, 演员可以自我调整,相关部门的负责人灯光也可以实时的调整,并最终确定

如果这是Virtual Production,那么在拍摄完之后,那就要进入后期特效制作的阶段了。 


要有一套完整的Previz系统,需要的东西有很多,例如绿屏,摄像机,显示器等等,但这些都不是我们关注的重点。Previz最重要的特点是能够把几个通道的内容实时的整合起来,类似于一个Hub,这几个通道的信息才是我们最关心,这些信息包括:
- 摄像机的运动信息
- 虚拟场景
- 真实的场景,既包括绿屏背景下的真实拍摄画面,也包括依靠运动捕捉采集的实时角色动画 

最后,便是这个Hub,说到底这就是一个实时的抠图和渲染的机箱。最困难的有两个,一个是如何获取摄像机的数据,包括位置、旋转角度、焦距等等,另一个是如何实时的合成。数据流才是我们最关心的东西。

\subsection*{如何获取摄像机信息?}
获取摄像机  信息的主流的方法分为三种,一种是固定摄像机在机械臂上,一种是通过图像处理的方法,还有一种是通过感应器的方式(Autodesk whitepaper 图)

图像处理 Survey, 图像处理的算法大概讲解一下, 给出marker的样子和实际拍摄的例子


这叫做Motion rig,实际上就是一个机器人手臂系统,所以摄像机的运动数据是非常准确的,而通过图像处理的方法,和算法有关系,而同时又涉及到靶标的布局等,精度是有问题的,而同时也由于布局的问题,要重新建立场景是比较困难的,由于不同光照的影响也非常大,搬到室外可能就问题比较大。但是通过图像定位的方法好处就是即使手持摄像机,也可以定位

\subsection*{如何合成最终画面?}
另外的一个难题就是如何把拍摄的真实视频和虚拟的场景结合起来。keying, 一方面就是要把绿屏替换掉,这个有很多成熟的图像处理的算法都可以实现,另一方面,就是要把 compositing 


坐标标定,原点和尺度


渲染的话,也分为两种,一种是要实时的展现在导演面前,另外一种是针对小成本制作,Shader写作


渲染出高精度的,这个时候就要利用到Nuke/Fusion/Smoke等合成软件了 

\subsection*{运动捕捉}

运动捕捉的数据,不同的帧速情况,要做插值,三种不同的系统优缺点。Motion Builder/自主研发的 (.detail facial capture\ldots手指的运动)如果角色还是虚拟角色的话,面部表情还要考虑lip sync



